Discuss The field of haptics is a very wide one, with many, many applications for many different purposes. Specifically, there are several concepts for wearable haptic devices that use vibration and motion to give information to the wearer. One early concept is Martin Frey’s CabBoots. These are shoes with an “integrated Guidance System” that is meant to guide users back onto whatever path or trail they are walking. It does this by using servos and motors to simulate how the human body steers itself back to the middle of a path when it nears the edge of said path [reference]. While this is somewhat similar in concept to what the [product name] is trying to do, the specific application of this form of haptics is not what the [product name] provides (the experience of “feeling music”). Another early concept is the Haptic Radar. The Haptic Radar is a wearable system composed of an array of “optical-hair modules” that would use invisible lasers to sense the surrounding environment. This information would then be translated by the system, and any surrounding dangers are communicated to the wearer, via vibrations on the wearer’s skin  where the danger was sensed [reference]. This is closer to what the [product name] provides than the CabBoots, but it still doesn’t use information from sound the way that the [product name] does. Two other devices, the hBracelet[reference] and the Wolverine[reference] are wearable haptic devices that are meant to simulate physical sensation, but in different ways. The hBracelet is meant to simulate the physical sensations of touch one would feel if they were actually in the virtual environment. The Wolverine (named for it’s appearance) is a wearable system that is meant to simulate the sensation of gripping rigid objects. Both of these devices aim to translate virtual events into physical sensations, but neither one translates sound into physical sensation, like the [product name]. Finally, the SUBPAC[reference] is one implementation of the haptic vest concept. It is meant for use as both a recreational item (for all users) and as a professional tool (for DJs, sound checkers, stage hands, etc.). However, the SUBPAC system is not cheap, making it somewhat more inaccessible to the general public. With the [product name], we hope to provide that same functionality, but at a reduced cost to the consumer.